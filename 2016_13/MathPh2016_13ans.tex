\documentclass[11pt]{jsarticle}
\setlength{\textheight}{43\baselineskip}
\addtolength{\textheight}{\topskip}
\setlength{\textheight}{21\footskip}
\setlength{\textwidth}{42zw}
\setlength{\hoffset}{+1zw}
\usepackage{amsmath,amssymb,bm}
\usepackage{fancyhdr} %ヘッダー
\usepackage{comment} 
\pagestyle{fancy}
\rhead{物理数学演習 \ 2016/05/19}
\renewcommand{\headrulewidth}{0pt} %ヘッダーラインを打ち出さない

\def \bun#1#2{\left(\frac{#1}{#2}\right)} %括弧付き分数マクロ
\def \vec#1{\mbox{\boldmath $#1$}} %ベクトルマクロ
\def \rot{\nabla \times} %ローテーション
\def \div{\nabla \cdot} %ダイバージェンス
\def \intt{\int\!\!\!\int} %2重積分
\def \intf#1#2#3#4{\int_{#1}^{#2}\!\!\!\int_{#3}^{#4}} %2重積分範囲付き
\def \inttt{\int\!\!\!\int\!\!\!\int} %3重積分
\def \intff#1#2#3#4#5#6{\int_{#1}^{#2}\!\!\!\int_{#3}^{#4}\!\!\!\int_{#5}^{#6}} %3重積分範囲付き

\usepackage{graphicx}
\usepackage{wrapfig}%テキスト回り込み

\def \Vec#1{\mbox{\boldmath $#1$}} %ベクトルマクロ

\begin{document}
\begin{center}
{\Large
演習問題その13  解答例}
\end{center}
%%%%
\begin{enumerate}


\item 次の関数をフーリエ変換せよ.
\begin{enumerate}
\item[(1)] $\delta (x)$\\
(解答例)
\begin{eqnarray*}
\mathcal{F}[\delta(x)]&=&\frac{1}{\sqrt{2\pi}}\int_{-\infty}^{\infty} \delta(x) e^{-ik x}dx\\
&=&\frac{1}{\sqrt{2\pi}}e^{-ik\cdot 0}=\frac{1}{\sqrt{2\pi}}
\end{eqnarray*}

\item[(2)] $\sin{k_0}x\qquad (ただしk_0は定数)$\\
(解答例)\\
 $\sin k_{0}x=(e^{ik_{0}x}-e^{-ik_{0}x})/2i$であるから
 \begin{eqnarray*}
 \mathcal{F}[\sin k_{0}x]&=&\mathcal{F} \left[ \frac{e^{ik_{0}x}-e^{-ik_{0}x}}{2i} \right]\\
 &=&\frac{1}{\sqrt{2\pi}}\int_{-\infty}^{\infty} \frac{e^{ik_{0}x}-e^{-ik_{0}x}}{2i} e^{-ik x}dx\\
 &=&\frac{1}{\sqrt{2\pi}}\frac{1}{2i}  \left\{  \int_{-\infty}^{\infty}1 \cdot e^{-i(k-k_{0})x}dx - \int_{-\infty}^{\infty}1 \cdot e^{-i(k+k_{0})x}dx\right\}\\
 &=&\frac{1}{\sqrt{2\pi}}\frac{1}{2i}\left( 2\pi\delta(k-k_{0})- 2\pi\delta(k+k_{0})\right)\\
 &=&\frac{i\pi}{\sqrt{2\pi}} (\delta(k+k_{0})-\delta(k-k_{0}))
 \end{eqnarray*}

\item[(3)] $\theta (x)$\\
ただし、$\theta (x)$はヘビサイド関数;\begin{equation*}
 \theta (x) =
   \begin{cases}
    0 \ (x<0)\\
    1 \ (x>0).
  \end{cases}
\end{equation*}
(解答例)
\begin{eqnarray*}
\mathcal{F}[\theta(x)]
&=&\frac{1}{\sqrt{2\pi}}\int_{-\infty}^{\infty} \theta(x) e^{-ik x}dx\\
&=&\frac{1}{\sqrt{2\pi}}\int_{0}^{\infty} e^{-ik x}dx = \frac{1}{\sqrt{2\pi}}\left[ \frac{1}{-ik}e^{-ikx} \right]_0^{\infty}\\
&=& \frac{1}{ik\sqrt{2\pi}}
\end{eqnarray*}


\end{enumerate}

\newpage
\item (応用問題)誘電体は電場をかけると分極する.誘電体に加える電場を$E(t)$とすると,分極ベクトルの大きさ$P(t)$は一般に
\begin{equation*}
P(t)=\int_{-\infty}^{\infty} \chi (t-t') E(t')dt'
\end{equation*}
とかける.ただし,$\chi$は帯電率と呼ばれ,次のようにかける.
\begin{equation*}
\chi(t)=
  \begin{cases}
    \chi_{0}e^{-t/\tau_{0}} \ (t \geq 0)\\
    0 \ (t<0)
  \end{cases}
\end{equation*}
ここで,$\chi_{0}, \tau$は定数である.電場$E(t)$を以下の形で加えたとき,$P(t)$がどのような振る舞いをするかフーリエ変換を用いて求める.
\begin{equation*}
E(t)=
  \begin{cases}
    E_{0}e^{-\epsilon t} \ (t \geq 0, \epsilon \rightarrow 0_{+})\\
    0 \ (t<0)
  \end{cases}
\end{equation*}
ここで,$\epsilon \rightarrow 0_{+}$なので,$t=0$でほとんどヘビサイド関数的に電場を加えたことに相当する($E(t)=E_{0} \theta (t)$).以下の問いに答えよ.\\

\begin{enumerate}
\item[(1)] $\chi (t),E(t)のフーリエ変換\mathcal{F}[\chi (t)],\mathcal{F}[E(t)] を求めよ.~また\mathcal{F}[e^{-t/\tau}],\mathcal{F}[\theta (t)]を求めよ.$\\
(解答例)
\begin{eqnarray*}
\mathcal{F}[\chi(t)]&=&\frac{1}{\sqrt{2\pi}}\int_{-\infty}^{\infty} \chi(t)e^{-ik t}dt=\frac{\chi_0}{\sqrt{2\pi}}\int_{0}^{\infty}e^{-t/\tau}e^{-ik t}dt\\
&=&\frac{\chi_0}{\sqrt{2\pi}}\int_{0}^{\infty}e^{-\left( \frac{1}{\tau}+ik\right)t}dt=\frac{1}{\sqrt{2\pi}}\frac{\chi_{0}}{ \frac{1}{\tau}+ik}
\end{eqnarray*}
これは、$1/\sqrt{2\pi}(1/\tau+ik)$が$e^{-t/\tau}$のフーリエ変換$\mathcal{F}[e^{-t/\tau}]=1/\sqrt{2\pi}(1/\tau+ik)$であることを示している。また、
\begin{eqnarray*}
\mathcal{F}[E(t)]&=&\frac{1}{\sqrt{2\pi}}\int_{-\infty}^{\infty} E(t)e^{-ik t}dt=\frac{E_0}{\sqrt{2\pi}}\int_{0}^{\infty}e^{-\epsilon t}e^{-ikt}dt\\
&=&\frac{E_0}{\sqrt{2\pi}}\int_{0}^{\infty} e^{-(\epsilon+ik)t}dt=\frac{1}{\sqrt{2\pi}}\frac{E_{0}}{\epsilon+ik} \rightarrow \frac{1}{\sqrt{2\pi}}\frac{E_{0}}{ik} \ (\epsilon \rightarrow 0_{+})
\end{eqnarray*}
これは、$1/ik\sqrt{2\pi}$が階段関数$\theta(t)$のフーリエ変換$\mathcal{F}[\theta(t)]=1/ik\sqrt{2\pi}$であることを示している。

\newpage
\item[(2)] フーリエ変換の合成積(たたみ込み積分)によって、$P(t)$のフーリエ変換は以下のようになる;
\begin{equation*}
\mathcal{F}[P(t)]=\sqrt{2\pi}\mathcal{F}[\chi (t)]\mathcal{F}[E(t)].
\end{equation*}
$P(t)$求めよ.\\
(ヒント:部分分数分解によって$\mathcal{F}[e^{-t/\tau}],\mathcal{F}[\theta (t)]$に分解する)\\
(解答例)
\begin{eqnarray*}
\mathcal{F}[P(t)]&=&\frac{1}{\sqrt{2\pi}}\frac{\chi_{0}E_{0}}{i\omega \left( \frac{1}{\tau}+i\omega \right)}=\frac{\chi_{0}E_{0}\tau}{\sqrt{2\pi}}\left( \frac{1}{i\omega} - \frac{1}{i\omega+\frac{1}{\tau}} \right)\\
&=&\chi_{0}E_{0}\tau \left\{ \mathcal{F}[\theta(t)]-\mathcal{F}[e^{-t/\tau}] \right\}\\
&=&\mathcal{F}\left[ \chi_{0}E_{0}\tau \left( \theta(t) -e^{-t/\tau} \right) \right]
\end{eqnarray*}
したがって
\begin{eqnarray*}
P(t)=\chi_{0}E_{0}\tau \left( \theta(t) -e^{-t/\tau} \right)
\end{eqnarray*}


\end{enumerate}

\newpage
\item 2変数関数$u(x,y)$について、以下の偏微分方程式を解け.\\
\begin{description}
\item[(1)] $\displaystyle{\frac{{\partial}^2 u}{\partial x \partial y}=x+y}$\\
\\
(解答例)\\
まず$y$で積分する.$x$のみに依存する任意関数$f(x)$を用いて,
\begin{eqnarray*}
\frac{\partial^2 u}{\partial x \partial y} = x+y &\Leftrightarrow&
\frac{\partial u}{\partial x } = xy+\frac{1}{2}y^2 + f(x) 
\end{eqnarray*}
さらに$x$で積分する.$f(x)$を$x$で積分しても$x$の関数であることに変わりはないので,
$\mathrm{d}g(x)/\mathrm{d}x=f(x)$となる$g(x)$を用いて,
\begin{eqnarray*}
u(x,y) = \frac{1}{2}xy(x +y) + g(x) + h(y).
\end{eqnarray*}
ここで$g(x)$は$x$のみに,$h(y)$は$y$のみに依存する任意関数.\\

\item[(2)] $\displaystyle{\frac{{\partial}^2 u}{\partial x^2}=x+y}$\\
\\
(解答例)
\begin{eqnarray*}
\frac{\partial^2 u}{\partial x^2} = x+y &\Leftrightarrow&
\frac{\partial u}{\partial x } = \frac{1}{2}x^2 + xy + f(y) \\
&\Leftrightarrow& u(x,y) =
\frac{1}{6}x^3 + \frac{1}{2} x^2 y + f(y)x + g(y).
\end{eqnarray*}
ただし$f(y),\ g(y)$は$y$のみに依存する関数.
\end{description}

\newpage
\item 以下の問いに答えよ.\\
\begin{description}
\item[(1)] $\displaystyle{\frac{\partial  u}{\partial x}=xy}$を満たし、$x=0$で$u=e^{-y}$に従って指数関数的に減少する解を求めよ.\\
(解答例)\\
\begin{eqnarray*}
\frac{\partial u}{\partial x}&=&xy\\
\Leftrightarrow u &=& \frac{1}{2}x^2y+f(y)
\end{eqnarray*}
$x=0$における条件式を代入すると、
\begin{eqnarray*}
e^{-y} &=& f(y)\\
u &=& \frac{1}{2}x^2y+e^{-y}.
\end{eqnarray*}

\item[(2)] $u(x,y)$についての偏微分方程式
$\displaystyle{\frac{\partial^2u}{\partial x \partial y}=10x^4}$ を解け.\\
ただし $u(x,0)=x^2, \ u(0,y)=e^y-1$ とする.\\
(解答例)\\
$x,y$で順に積分していくと,
\begin{eqnarray*}
\frac{\partial u}{\partial y} &=& 2x^5+f(y). \\
u &=& 2x^5y+g(y)+h(x). \;\;\;\; \hspace{0.5cm} \left(f=\frac{dg}{dy}\right)
\end{eqnarray*}
条件式より,
\begin{eqnarray}
u(x,0)=g(0)+h(x)=x^2 \;\;\;\; \Rightarrow \;\;\;\; h(x) &=& x^2-g(0).
\label1 \\
u(0,y)=g(y)+h(0)=e^y-1 \;\;\;\; \Rightarrow \;\;\;\; g(y) &=& e^y-1-h(0) =
e^y+g(0)-1. \;\;\;\; (\because (1)式) \nonumber
\end{eqnarray}
\begin{eqnarray*}
\therefore \ g(y)+h(x) = x^2+e^y-1. 
\end{eqnarray*}
よって $\;\;\; u(x,y)=2x^5y+x^2+e^y-1$.
\end{description}

\newpage
\item
$\displaystyle{y\frac{\partial^2u}{\partial y^2}=
\frac{\partial u}{\partial y}}$を満たす$u(x,y)$のうち,
$y=\pm1$で$u(x,y)=0,\ y=0$で$u=x^2$となるものを求めよ.\\
(解答例)\\
$u=X(x)Y(y)$とおく. 与式に代入すると,
\\
\[
yX(x)\frac{\mathrm{d}^2Y(y)}{\mathrm{d}y^2}=
X(x)\frac{\mathrm{d}Y(y)}{\mathrm{d}y}.
\]
ここで$X(x)$は両辺にかかっているので, 任意の関数$f(x)$である.
両辺を$X$で割って,
%
\begin{eqnarray*}
y\frac{\mathrm{d}^2Y(y)}{\mathrm{d}y^2} = \frac{\mathrm{d}Y(y)}{\mathrm{d}y}.
\end{eqnarray*}
%
ここで$\displaystyle{\frac{\mathrm{d}Y}{\mathrm{d}y}=p}$とおくと,
\begin{eqnarray*}
&& yp' = p \ \Leftrightarrow \
\frac1p\frac{\mathrm{d}p}{\mathrm{d}y} = \frac1y.  \\[5pt]
&\therefore& \ln|p| = \ln|y|+C'. \hspace{1cm} (C'は積分定数) \\[7pt]
&\Leftrightarrow& \frac{\mathrm{d}Y}{\mathrm{d}y}=p = Cy. \hspace{1cm}
\left(C=\pm e^{C'}\right) \\[3pt]
&\therefore& \ Y = C_1y^2+C_2. \hspace{1cm} \left(C_1=\frac C2 とした \right)\\
\end{eqnarray*}
よって $u(x,y)=f(x)y^2+g(x)$.
境界条件より$f(x),\ g(x)$を求めると,
 \\[5pt]
$y=\pm1 で u=0$ \ より \ $f(x)=-g(x)$.\\[5pt]
また, $y=0 で u=x^2$ \ より \ $x^2=g(x)$.\\[5pt]
よって求める解は \ $u(x,y)=-x^2y^2+x^2$.


\newpage
\item (超重要!) 熱伝導方程式$\displaystyle{\frac{\partial u}{\partial t} =
\kappa \frac{\partial^2 u}{\partial x^2}~(\kappa >0)}$を変数分離法で解け.\\
(解答例)\\[8pt]
$u=T(t)X(x)$と変数分離し、
\begin{eqnarray*}
&&X\frac{d T}{d t} = \kappa T \frac{d^2 X}{d x^2} \\[5pt]
&\Leftrightarrow& \frac{1}{\kappa T}\frac{d T}{d t} =
\frac{1}{X} \frac{d^2 X}{d x^2} = \omega
\end{eqnarray*}
と定数$\omega$で置く.
これを解くと、\\[12pt]
%
(i) $\omega=0$ の場合
%
\begin{eqnarray*}
T &=& C_1. \\
X &=& C_2x+C_3. \\
\therefore \;\;\; u(x,t) &=& Ax+B.
\end{eqnarray*}
%
(ii) $\omega<0$ の場合
%
\begin{eqnarray*}
T &=& C_1e^{\kappa \omega t}. \\
X &=& C_2\sin\sqrt{-\omega}x+C_3\cos\sqrt{-\omega}x. \\
\therefore \;\;\; u(x,t) &=&
e^{\kappa \omega t}\left(A\sin\sqrt{-\omega}x+B\cos\sqrt{-\omega}x\right).
%
\end{eqnarray*}
%
(iii) $\omega>0$ の場合
%
\begin{eqnarray*}
 T &=& C_1 e^{\kappa\omega t}.\\
 X &=& C_2 e^{\sqrt\omega x} + C_3 e^{-\sqrt\omega x}.\\
 \therefore \ u &=&
e^{\kappa\omega t} \left(A e^{\sqrt\omega x} + B e^{-\sqrt\omega x} \right).
\end{eqnarray*}
%
$C_1,C_2,C_3$は積分定数.また、$C_1C_2=A, \ C_1C_3=B$とした.\\

\newpage
\item (超重要!) 波動方程式
\begin{eqnarray*}
\frac{\partial^2 u}{\partial t^2}  = c^2 \frac{\partial^2 u}{\partial x^2}
\end{eqnarray*}
を変数分離法を用いて解け.\\
(解答例)\\[8pt]
$u(x,t)=X(x)T(t)$とおく(変数分離).このとき、
%
\begin{eqnarray*}
X\frac{\mathrm{d}^2 T}{\mathrm{d} t^2} =
c^2T \frac{\mathrm{d}^2 X}{\mathrm{d} x^2} \
\Leftrightarrow \ \frac{1}{T}\frac{\mathrm{d}^2 T}{\mathrm{d} t^2}=
c^2 \frac{1}{X}\frac{\mathrm{d}^2 X}{\mathrm{d} x^2}. \quad \cdots (a)
\end{eqnarray*}
%
左辺は$t$のみに、右辺は$x$のみに依存し、これらが$x,t$の値によらず等しいので、$(a)$式の両辺はていすうである.
この定数を$\lambda$とおくと、\\
%
(i) $\lambda=0$のとき
%
\begin{eqnarray*}
\frac{1}{T}\frac{\mathrm{d}^2 T}{\mathrm{d} t^2}  = 0 \ 
&\Leftrightarrow& \ T = C_1t+C_2. \\[5pt]
c^2 \frac{1}{X}\frac{\mathrm{d}^2 X}{\mathrm{d} x^2} = 0 \ 
&\Leftrightarrow& \ X = C_3x+C_4. \\
\\
\therefore \hspace{0.5cm} u(x,t) &=& (C_1t+C_2)(C_3x+C_4).
\end{eqnarray*}
%
(ii) $\lambda<0$ のとき
%
\begin{eqnarray*}
\frac{1}{T}\frac{\mathrm{d}^2 T}{\mathrm{d} t^2} = \lambda<0 \ 
&\Leftrightarrow& \ T = C_1\sin\sqrt{-\lambda}t+C_2\cos\sqrt{-\lambda}t.\\[5pt]
c^2 \frac{1}{X}\frac{\mathrm{d}^2 X}{\mathrm{d} x^2} = \lambda<0 \ 
&\Leftrightarrow& \ X =
C_3\sin\frac{\sqrt{-\lambda}}cx+C_4\cos\frac{\sqrt{-\lambda}}cx.\\[5pt]
\therefore \hspace{0.5cm} u(x,t) &=& 
\left(C_1\sin\sqrt{-\lambda}t+C_2\cos\sqrt{-\lambda}t\right)
\left(C_3\sin\frac{\sqrt{-\lambda}}cx+C_4\cos\frac{\sqrt{-\lambda}}cx\right).
\end{eqnarray*}
%
(iii) $\lambda>0$ のとき
%
\begin{eqnarray*}
\frac{1}{T}\frac{\mathrm{d}^2 T}{\mathrm{d} t^2}  = \lambda>0 \
&\Leftrightarrow& \ T = C_1e^{\sqrt{\lambda}t}+C_2e^{-\sqrt{\lambda}t} \\[5pt]
c^2 \frac{1}{X}\frac{\mathrm{d}^2 X}{\mathrm{d} x^2} = \lambda>0 \
&\Leftrightarrow& \ X = C_3e^{\frac{\sqrt{\lambda}}cx}+
C_4e^{-\frac{\sqrt{\lambda}}cx} \\[5pt]
\therefore \hspace{0.5cm} u(x,t) &=&
\left(C_1e^{\sqrt{\lambda}t}+C_2e^{-\sqrt{\lambda}t}\right)
\left(C_3e^{\frac{\sqrt{\lambda}}cx}+C_4e^{-\frac{\sqrt{\lambda}}cx}\right).
\end{eqnarray*}

\newpage
\item 2次元極座標におけるラプラス方程式
\begin{eqnarray*}
 \Delta f &=&  \frac{1}{r} \frac{\partial }{\partial r} \left(r \frac{\partial f}{\partial r}\right)+ \frac{1}{r^2}\frac{\partial^2 f}{\partial \phi^2}=0 
\end{eqnarray*}
を$f(r,\phi)=R(r)\Phi(\phi)$とする変数分離によって解け.\\
(ヒント:$r$の積分で困ったら過去の演習の「同次形微分方程式」を参照)\\
(解答例)\\[8pt]
$f(r,\ \phi)= R(r) \Phi(\phi)$より
\begin{eqnarray*}
\Delta f =
\frac{1}{r} \frac{\partial }{\partial r}
\left(r \frac{\partial (R\Phi)}{\partial r}\right)+
\frac{1}{r^2}\frac{\partial^2 (R\Phi)}{\partial \phi^2} &=& 0.\\
\Leftrightarrow \ \frac{\Phi}{r} \frac{\mathrm{d}}{\mathrm{d}r}
\left(r \frac{\mathrm{d} R}{\mathrm{d} r}\right)+
\frac{R}{r^2}\frac{\mathrm{d}^2 \Phi}{\mathrm{d} \phi^2} &=& 0.\\
\Leftrightarrow \ \frac{r}{R} \frac{\mathrm{d} }{\mathrm{d} r}
\left(r \frac{\mathrm{d} R}{\mathrm{d} r}\right)+
\frac{1}{\Phi}\frac{\mathrm{d}^2 \Phi}{\mathrm{d} \phi^2} &=& 0.
\end{eqnarray*}
第1項は$r$のみに、第2項は$\phi$のみに依存し、これらが$r,\phi$の値によらず等しいので、どちらも定数である.
この定数を$\lambda$とおくと、
\[
\frac rR\frac {\mathrm{d}}{\mathrm{d}r}
\left(r\frac{\mathrm{d}R}{\mathrm{d}r}\right)=
-\frac1\Phi\frac{\mathrm{d}^2\Phi}{\mathrm{d}\phi^2}=\lambda.
\]
(i)$\lambda=0$の場合、\\
(第1項)$=0$から、
\begin{eqnarray*}
\frac{r}{R} \frac{\mathrm{d} }{\mathrm{d} r}
\left(r \frac{\mathrm{d} R}{\mathrm{d} r}\right) &=& 0
\  \Leftrightarrow \  r \frac{\mathrm{d} R}{\mathrm{d} r} =
C_1=\mathrm{const.}  \\[5pt]
\Leftrightarrow \ R &=& C_1\ln{(r)} + C_2. \quad ( C_2=\mathrm{const.}. ) \\
\end{eqnarray*}
%
(第2項)$=0$から、
\begin{eqnarray*}
\frac{1}{\Phi}\frac{\mathrm{d}^2 \Phi}{\mathrm{d} \phi^2} &=& 0
\  \Leftrightarrow \   \frac{\mathrm{d} \Phi}{\mathrm{d} \phi} =
C_3=\mathrm{const.}  \\[5pt]
\Leftrightarrow \ \Phi &=& C_3 \phi + C_4.\quad ( C_4=\mathrm{const.}. ) \\
\end{eqnarray*}
%
以上より
\begin{eqnarray*}
f= \left( C_1\ln{(r)} + C_2 \right) \left( C_3 \phi + C_4 \right).
\end{eqnarray*}
%
(ii)\ $\lambda<0$の場合
%
\begin{eqnarray*}
\frac{1}{\Phi}\frac{\mathrm{d}^2 \Phi}{\mathrm{d} \phi^2} &=&
-\lambda>0 \\[5pt]
\Leftrightarrow \ \Phi &=& C_3 e^{\sqrt{-\lambda} \phi} +
C_4 e^{-\sqrt{-\lambda} \phi}  \hspace{0.5cm} (C_n=\mathrm{const.}
\hspace{0.2cm}(n=1,\ 2,\ 3,\ 4))
\end{eqnarray*}
また、
%
\begin{eqnarray*}
\frac{r}{R} \frac{\mathrm{d} }{\mathrm{d} r}
\left(r \frac{\mathrm{d} R}{\mathrm{d} r}\right) &=& \lambda <0 \\[8pt]
\therefore \ r^2R''+rR'-\lambda R &=& 0
\end{eqnarray*}
%
これは$r=e^u$とおくと($\mathrm{d}r=e^u\mathrm{d}u$)
%
\begin{eqnarray*}
\frac{1}{R} \frac{\mathrm{d} }{\mathrm{d} u}
\left( \frac{\mathrm{d} R}{\mathrm{d} u}\right) &=& \lambda<0 \\[8pt]
\Leftrightarrow \ R &=& C_1 e^{i\sqrt{-\lambda} u}
+ C_2 e^{-i\sqrt{-\lambda} u} =
C_1 r^{i\sqrt{-\lambda} } + C_2 r^{-i\sqrt{-\lambda} }.
\end{eqnarray*}
よって、
\begin{eqnarray*}
 f = \left(C_1 r^{i\sqrt{-\lambda} } + C_2 r^{-i\sqrt{-\lambda} } \right)
\left( C_3 e^{\sqrt{-\lambda} \phi}+
C_4 e^{-\sqrt{-\lambda} \phi} \right).\\
\end{eqnarray*}
%
(iii)$\lambda>0$の場合
%
\begin{eqnarray*}
&&\frac{1}{\Phi}\frac{\mathrm{d}^2 \Phi}{\mathrm{d} \phi^2} = -\lambda<0 \\[5pt]
&\Leftrightarrow& \ \Phi = C_3 \sin\sqrt\lambda \phi +
C_4 \cos\sqrt\lambda \phi.
\end{eqnarray*}
%
また、
%
\begin{eqnarray*}
\frac{r}{R} \frac{\mathrm{d} }{\mathrm{d} r}
\left(r \frac{\mathrm{d} R}{\mathrm{d} r}\right) &=& \lambda>0.
\end{eqnarray*}
%
これは$r=e^u$とおくと($\mathrm{d}r=e^u\mathrm{d}u$)
\begin{eqnarray*}
&&\frac{1}{R} \frac{\mathrm{d} }{\mathrm{d} u}
\left( \frac{\mathrm{d} R}{\mathrm{d} u}\right) = \lambda>0 \\[5pt]
&\Leftrightarrow& \ R =
C_1 e^{\sqrt{\lambda} u} + C_2 e^{-\sqrt{\lambda} u} =
C_1 r^{\sqrt\lambda } + C_2 r^{-\sqrt\lambda }.
\end{eqnarray*}
よって、
\begin{eqnarray*}
 f = \left(C_1 r^{\sqrt\lambda } + C_2 r^{-\sqrt\lambda } \right)
\left(C_3 \sin\sqrt\lambda \phi + C_4 \cos\sqrt\lambda \phi \right).
\end{eqnarray*}

\end{enumerate}
\end{document}