\documentclass[11pt]{jsarticle}
\setlength{\textheight}{43\baselineskip}
\addtolength{\textheight}{\topskip}
\setlength{\textheight}{21\footskip}
\setlength{\textwidth}{42zw}
\setlength{\hoffset}{+1zw}
\usepackage{amsmath,amssymb,bm}
\usepackage{fancyhdr} %ヘッダー
\usepackage{comment} 
\pagestyle{fancy}
\rhead{物理数学演習 \ 2016/05/19}
\renewcommand{\headrulewidth}{0pt} %ヘッダーラインを打ち出さない

\def \bun#1#2{\left(\frac{#1}{#2}\right)} %括弧付き分数マクロ
\def \vec#1{\mbox{\boldmath $#1$}} %ベクトルマクロ
\def \rot{\nabla \times} %ローテーション
\def \div{\nabla \cdot} %ダイバージェンス
\def \intt{\int\!\!\!\int} %2重積分
\def \intf#1#2#3#4{\int_{#1}^{#2}\!\!\!\int_{#3}^{#4}} %2重積分範囲付き
\def \inttt{\int\!\!\!\int\!\!\!\int} %3重積分
\def \intff#1#2#3#4#5#6{\int_{#1}^{#2}\!\!\!\int_{#3}^{#4}\!\!\!\int_{#5}^{#6}} %3重積分範囲付き

\usepackage{graphicx}
\usepackage{wrapfig}%テキスト回り込み

\def \Vec#1{\mbox{\boldmath $#1$}} %ベクトルマクロ

\begin{document}
\begin{center}
{\Large
演習問題その13  フーリエ変換(3),偏微分方程式(1)}
\end{center}
%%%%
※以下、$y'=\frac{dy}{dx},~y''=\frac{d^2 y}{dx^2}$とする.
\begin{enumerate}

\item 次の関数をフーリエ変換せよ.
\begin{enumerate}
\item[(1)] $\delta (x)$
\item[(2)] $\sin{k_0}x\qquad (ただしk_0は定数)$
\item[(3)] $\theta (x)$\\
ただし、$\theta (x)$はヘビサイド関数;\begin{equation*}
 \theta (x) =
   \begin{cases}
    0 \ (x<0)\\
    1 \ (x>0).
  \end{cases}
\end{equation*}
\end{enumerate}

\newpage
\item (応用問題)誘電体は電場をかけると分極する.誘電体に加える電場を$E(t)$とすると,分極ベクトルの大きさ$P(t)$は一般に
\begin{equation*}
P(t)=\int_{-\infty}^{\infty} \chi (t-t') E(t')dt'
\end{equation*}
とかける.ただし,$\chi$は帯電率と呼ばれ,次のようにかける.
\begin{equation*}
\chi(t)=
  \begin{cases}
    \chi_{0}e^{-t/\tau_{0}} \ (t \geq 0)\\
    0 \ (t<0)
  \end{cases}
\end{equation*}
ここで,$\chi_{0}, \tau$は定数である.電場$E(t)$を以下の形で加えたとき,$P(t)$がどのような振る舞いをするかフーリエ変換を用いて求める.
\begin{equation*}
E(t)=
  \begin{cases}
    E_{0}e^{-\epsilon t} \ (t \geq 0, \epsilon \rightarrow 0_{+})\\
    0 \ (t<0)
  \end{cases}
\end{equation*}
ここで,$\epsilon \rightarrow 0_{+}$なので,$t=0$でほとんどヘビサイド関数的に電場を加えたことに相当する($E(t)=E_{0} \theta (t)$).以下の問いに答えよ.\\

\begin{enumerate}
\item[(1)] $\chi (t),E(t)のフーリエ変換\mathcal{F}[\chi (t)],\mathcal{F}[E(t)] を求めよ.~また\mathcal{F}[e^{-t/\tau}],\mathcal{F}[\theta (t)]を求めよ.$

\newpage
\item[(2)] フーリエ変換の合成積(たたみ込み積分)によって、$P(t)$のフーリエ変換は以下のようになる;
\begin{equation*}
\mathcal{F}[P(t)]=\sqrt{2\pi}\mathcal{F}[\chi (t)]\mathcal{F}[E(t)].
\end{equation*}
$P(t)$求めよ.\\
(ヒント:部分分数分解によって$\mathcal{F}[e^{-t/\tau}],\mathcal{F}[\theta (t)]$に分解する)
\end{enumerate}

\newpage
\item 2変数関数$u(x,y)$について、以下の偏微分方程式を解け.\\
\begin{description}
\item[(1)] $\displaystyle{\frac{{\partial}^2 u}{\partial x \partial y}=x+y}$\\
\item[(2)] $\displaystyle{\frac{{\partial}^2 u}{\partial x^2}=x+y}$
\end{description}

\newpage
\item 以下の問いに答えよ.\\
\begin{description}
\item[(1)] $\displaystyle{\frac{\partial  u}{\partial x}=xy}$を満たし、$x=0$で$u=e^{-y}$に従って指数関数的に減少する解を求めよ.
\item[(2)] $u(x,y)$についての偏微分方程式
$\displaystyle{\frac{\partial^2u}{\partial x \partial y}=10x^4}$ を解け.\\
ただし $u(x,0)=x^2, \ u(0,y)=e^y-1$ とする.
\end{description}

\newpage
\item
$\displaystyle{y\frac{\partial^2u}{\partial y^2}=
\frac{\partial u}{\partial y}}$を満たす$u(x,y)$のうち,
$y=\pm1$で$u(x,y)=0,\ y=0$で$u=x^2$となるものを求めよ.

\newpage
\item (超重要!) 熱伝導方程式$\displaystyle{\frac{\partial u}{\partial t} =
\kappa \frac{\partial^2 u}{\partial x^2}~(\kappa >0)}$を変数分離法で解け.

\newpage
\item (超重要!) 波動方程式
\begin{eqnarray*}
\frac{\partial^2 u}{\partial t^2}  = c^2 \frac{\partial^2 u}{\partial x^2}
\end{eqnarray*}
を変数分離法を用いて解け.

\newpage
\item 2次元極座標におけるラプラス方程式
\begin{eqnarray*}
 \Delta f &=&  \frac{1}{r} \frac{\partial }{\partial r} \left(r \frac{\partial f}{\partial r}\right)+ \frac{1}{r^2}\frac{\partial^2 f}{\partial \phi^2}=0 
\end{eqnarray*}
を$f(r,\phi)=R(r)\Phi(\phi)$とする変数分離によって解け.\\
(ヒント:$r$の積分で困ったら過去の演習の「同次形微分方程式」を参照)

\end{enumerate}
\end{document}