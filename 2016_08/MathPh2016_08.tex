\documentclass[a4paper,11pt,fleqn]{jarticle}
\usepackage[dvipdfmx]{graphicx}
\usepackage{float}
\usepackage{amsmath}
\usepackage{fancyhdr}

\def \vec#1{\mbox{\boldmath $#1$}} %ベクトルマクロ
\def \bun#1#2{\left(\frac{#1}{#2}\right)} %括弧つき分数マクロ
\def \rot{\nabla \times} %rot
\def \div{\nabla \cdot} %div
\def \intt{\int\!\!\!\int} %2重積分
\def \inttt{\int\!\!\!\int\!\!\!\int} %3重積分

% ページレイアウト
\setlength{\topmargin}{10mm}
  \addtolength{\topmargin}{-1in}
\setlength{\oddsidemargin}{30mm}
  \addtolength{\oddsidemargin}{-1in}
\setlength{\textwidth}{150mm}
\setlength{\textheight}{250mm}
\setlength{\headsep}{2zw}
\setlength{\headheight}{2zw}
\setlength{\topskip}{15mm}
\linespread{1.0}

% サブセクションを1.,2.にする設定
\renewcommand{\thesubsection}{\arabic{subsection}.}

% サブサブセクションを(1),(2)にする設定
\renewcommand{\thesubsubsection}{(\arabic{subsubsection})}

% 大問2の3番目の計算式のラベルを(2.3)にする設定
% 計算式の参照には\eqref{eq:hoge}を使う
\makeatletter
  \renewcommand{\theequation}{\arabic{subsection}.\arabic{equation}}
  \@addtoreset{equation}{subsection}
\pagestyle{fancy}
% ヘッダーの設定
  \lhead[物理数学 2016.04.28]{\leftmark}
  \rhead[\leftmark]{物理数学 2016.04.28}
\renewcommand{\headrulewidth}{0pt}
\makeatother



\begin{document}


\begin{center}
\begin{Large}
演習問題その8  常微分方程式(3)
\end{Large}
\end{center}

\subsection{}
次の1階高次の微分方程式を解け.ただし、$dy/dx=y'$とする.
\begin{eqnarray*}
y=xy'-y^2/4
\end{eqnarray*}

\newpage
\subsection{}
次の微分方程式の一般解を求めよ.
\subsubsection{$y''+4y=x$}

\vspace{80mm}
\subsubsection{$y''-5y'+6y=e^{-x}$}

\newpage
\subsection{}
次の高階線形微分方程式を解け.
\subsubsection{$y'''+y''-21y'-45y=0$}

\vspace{80mm}
\subsubsection{$y'''+y''-5y'+3y=0$}

\newpage
\subsection{}
次の微分方程式を解け.
\subsubsection{$y=x(y'+1)+y'$}

\vspace{90mm}
\subsubsection{$y=x{y'}^2+{y'}^2$}

\newpage
\subsection{}
$p=y'$とおいて階数の引き下げを行うことにより、次の微分方程式を解け.
\subsubsection{$xy''-3y'=0$}

\vspace{90mm}
\subsubsection{$x^2y''=2xy'+x^2$}

\newpage
\subsection{}
次の高階同次形の微分方程式を解け.
\subsubsection{$x^2y''+xy'+3y=0$}

\vspace{90mm}
\subsubsection{$yy''-{y'}^2-6xy^2=0$}

\newpage
\subsection{$+\alpha$問題}
\subsubsection{単振り子}
振り子の最下点からの円弧に沿った長さを$s$をすると、
単振り子の運動方程式は、
\begin{eqnarray*}
m\frac{d^2s}{dt^2}=-mg\sin\theta\sim-mg\frac{s}{l}
\end{eqnarray*}
ここで$\theta\ll1$である。ある時刻tにおける変位s(t)を求めよ。
ここで$t=0$のときの質点は最下点$s=0$で速度$v_0$とする。

\newpage
\subsubsection{強制振動}
\begin{eqnarray*}
\frac{d^2y}{dt^2}+\omega_0^2y=F\cos{\omega{t}}
\end{eqnarray*}
は固有角周波数$\omega_0$の振動子に外力$F\cos{\omega{t}}$を加えたときの振動子の運動を記述する方程式である。このとき、以下の問いに答えよ。
\begin{description}
\item[(1)]$\omega\neq\omega_0$の場合の微分方程式の一般解を求めよ。
\item[(2)]$\omega=\omega_0$の場合の微分方程式の一般解を求めよ。
\end{description}


\end{document}