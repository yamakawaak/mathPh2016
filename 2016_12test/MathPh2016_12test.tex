\documentclass[11pt]{jarticle}
\usepackage[dvipdfmx]{graphicx}
\usepackage{float}
\setlength{\textheight}{43\baselineskip}
\addtolength{\textheight}{\topskip}
\setlength{\textheight}{21\footskip}
\setlength{\textwidth}{42zw}
\setlength{\hoffset}{-5zw}
\usepackage{amsmath,amssymb,bm}
%
\usepackage{fancyhdr} %ヘッダー
\pagestyle{fancy}
\rhead{物理数学 \ 2016.05.19}
\renewcommand{\headrulewidth}{0pt} %ヘッダーラインを打ち出さない
%
\def \bun#1#2{\left(\frac{#1}{#2}\right)} %括弧付き分数マクロ
\def \vec#1{\mbox{\boldmath $#1$}} %ベクトルマクロ
\def \rot{\nabla \times} %ローテーション
\def \div{\nabla \cdot} %ダイバージェンス
\def \intt{\int\!\!\!\int} %2重積分
\def \intf#1#2#3#4{\int_{#1}^{#2}\!\!\!\int_{#3}^{#4}} %2重積分範囲付き
\def \inttt{\int\!\!\!\int\!\!\!\int} %3重積分
\def \intff#1#2#3#4#5#6{\int_{#1}^{#2}\!\!\!\int_{#3}^{#4}\!\!\!\int_{#5}^{#6}} %3重積分範囲付き
%
%%------------------------------%%
\begin{document}
\begin{center}
{\Large
演習問題その12小テスト}\\
\ \\
\underline{学籍番号:          }, 
\underline{氏名:            }
\end{center}
\begin{enumerate}
%%-----(問題開始)-----%%
\item[1.]関数$f(x)$のフーリエ変換を$F(k)$とする.以下の問いに答えよ.
\begin{enumerate}
\item[(1)] フーリエ変換とフーリエ逆変換の定義式を書け.
\item[(2)] $\mathcal{F}[e^{ix}]=\sqrt{2\pi}\delta (k-1)$を示せ.
\end{enumerate}

%%-----(問題終了)-----%%
\end{enumerate}


\newpage
%%------------------------------%%
%%-----(   解答ここから   )-----%%
\begin{center}
{\Large
演習問題その12小テスト 解答例}\\
\end{center}
\begin{enumerate}
\item[1.]関数$f(x)$のフーリエ変換を$F(k)$とする.以下の問いに答えよ.
\begin{enumerate}
\item[(1)] フーリエ変換とフーリエ逆変換の定義式を書け.\\
(解答例)\\
フーリエ変換は、$\displaystyle{\mathcal{F}[f(x)]=F(k)=\frac{1}{\sqrt{2\pi}}\int_{-\infty}^{\infty} f(x) e^{-ik x}dx}$\\
\\
フーリエ逆変換は、$\displaystyle{{\mathcal{F}}^{-1}[F(k)]=f(x)=\frac{1}{\sqrt{2\pi}}\int_{-\infty}^{\infty} F(k) e^{+ik x}dk}$\\
\\

\item[(2)] $\mathcal{F}[e^{ix}]=\sqrt{2\pi}\delta (k-1)$を示せ.
\\
(解答例)
フーリエ変換の定義式より、
\begin{eqnarray*}
\mathcal{F}[e^{ix}] &=& \frac{1}{\sqrt{2\pi}}\int_{-\infty}^{\infty} e^{ix}e^{-ikx}{\rm d}x\\
&=& \frac{1}{\sqrt{2\pi}}\int_{-\infty}^{\infty} e^{i(1-k)x}{\rm d}x
\end{eqnarray*}
デルタ関数の積分表示より、
\[\delta (k)=\frac{1}{2\pi}\int_{-\infty}^{\infty}e^{ikx}{\rm d}x\]
これを$k\rightarrow 1-k$と書き直すことで与式が得られる.\\

\newpage
(採点基準) 10点満点.\\
(1)フーリエ変換、逆変換それぞれ+3点.\\
(2)\\
$\mathcal{F}[e^{ix}]$が明記されていて+2点.\\
$\delta (k-1)$の表式が説明されていて+2点(いちおう証明問題なので最低限の説明がないものは減点した).\\
そのほか、オマケで加点してある.ちなみに(2)はフーリエ逆変換を用いても解くことができる.

\end{enumerate}

%%-----(解答終了)-----%%
\end{enumerate}

\newpage


\end{document}
