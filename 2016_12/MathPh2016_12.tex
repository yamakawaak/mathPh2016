\documentclass[11pt]{jsarticle}
\setlength{\textheight}{43\baselineskip}
\addtolength{\textheight}{\topskip}
\setlength{\textheight}{21\footskip}
\setlength{\textwidth}{42zw}
\setlength{\hoffset}{+1zw}
\usepackage{amsmath,amssymb,bm}
\usepackage{fancyhdr} %ヘッダー
\usepackage{comment} 
\pagestyle{fancy}
\rhead{物理数学演習 \ 2016/05/16}
\renewcommand{\headrulewidth}{0pt} %ヘッダーラインを打ち出さない

\def \bun#1#2{\left(\frac{#1}{#2}\right)} %括弧付き分数マクロ
\def \vec#1{\mbox{\boldmath $#1$}} %ベクトルマクロ
\def \rot{\nabla \times} %ローテーション
\def \div{\nabla \cdot} %ダイバージェンス
\def \intt{\int\!\!\!\int} %2重積分
\def \intf#1#2#3#4{\int_{#1}^{#2}\!\!\!\int_{#3}^{#4}} %2重積分範囲付き
\def \inttt{\int\!\!\!\int\!\!\!\int} %3重積分
\def \intff#1#2#3#4#5#6{\int_{#1}^{#2}\!\!\!\int_{#3}^{#4}\!\!\!\int_{#5}^{#6}} %3重積分範囲付き

\usepackage{graphicx}
\usepackage{wrapfig}%テキスト回り込み

\def \Vec#1{\mbox{\boldmath $#1$}} %ベクトルマクロ

\begin{document}
\begin{center}
{\Large
演習問題その12  フーリエ変換(2)}
\end{center}
%%%%
※以下、$y'=\frac{dy}{dx},~y''=\frac{d^2 y}{dx^2}$とする.
\begin{enumerate}

\item 次の関数をフーリエ級数に展開せよ.
\begin{equation*}
 f(x) =
   \begin{cases}
    -1 \ (-T<x<0)\\
    0 \ (x=0)\\
    1 \ (0<x<T)
  \end{cases}
\end{equation*}

\newpage
\item $f(x)=e^x(周期2\pi,-\pi<x<\pi)$を指数関数型のフーリエ級数に展開せよ.

\newpage
\item 以下の問に答えよ.
\begin{enumerate}
\item[(1)] \ 次の関数$f(x)$をフーリエ級数展開せよ(第11回演習問題参照).
\begin{eqnarray}
f(x) = x^2   (-\pi<x<\pi, \hspace{0.2cm} \mbox{周期}2\pi)\nonumber
\end{eqnarray}

\item[(2)]  区間$[-\pi,\pi]$で区分的に連続な関数$f(x)$のフーリエ係数に対して
\[
\frac{a_0^2}2+\sum_{n=1}^{\infty}(a_n^2+b_n^2)=\frac1{\pi}\int_{-\pi}^\pi\{f(x)\}^2\mathrm{d}x \;\;\;\;\; (パーセバルの等式)
\]
が成り立つことを示せ.

\newpage
\item[(3)]  \ (1),(2)の結果を用いて以下の等式を示せ.
\[
1+\frac1{2^4}+\frac1{3^4}+\frac1{4^4}+\cdots=\frac{\pi^4}{90}.
\]
\end{enumerate}


\vspace{80mm}
\item 次の積分値を計算せよ.
\begin{equation*}
\int_{-\infty}^{\infty} \delta (x^2-a^2)e^{ix} {\rm d}x
\end{equation*}

\newpage
\item デルタ関数の定義により次の関係式を証明せよ.
\begin{equation*}
x\delta '(x)+\delta (x)=0 \qquad (ヒント:部分積分を用いる)
\end{equation*}


\newpage
\item 2階の定数係数線形の微分方程式
\begin{eqnarray*}
\label{a}
y''+3y'+2y=e^{ix}
\end{eqnarray*}
の特殊解をフーリエ変換を用いて求める.以下の問に答えよ.
\begin{description}
\item[(1)] $\mathcal{F}[e^{ix}]=\sqrt{2\pi}\delta (k-1)$を示せ.
\item[(2)] 任意の関数$f(k)$に対して、$\displaystyle{\int_{-\infty}^{\infty} f(k)\delta (k-k_0){\rm d}k=f(k_0)}$を示せ.
\newpage
\item[(3)] $y(x)$のフーリエ変換を$Y(k)$として、与えられた微分方程式より、$Y(u)$を$u$とデルタ関数を用いて表せ.
\item[(4)] (3)で求めた$Y(k)$をフーリエ逆変換せよ.(得られる$y(x)$は与えられた微分方程式を満たす特殊解である)
\end{description}



\newpage
\section*{$+\alpha 問題$}
\item 次の積分値を計算せよ.\\
\begin{description}
\item[(1)] $\displaystyle{\int_{-\infty}^{\infty} \delta (2x-1)\sin{\pi x} {\rm d}x}$
\vspace{80mm}
\item[(2)] $\displaystyle{\int_{-\infty}^{\infty} \delta '(x)(x^2-x) {\rm d}x}$
\end{description}

\newpage
\item $\mathcal{F}[f(x)]=F(k),~\mathcal{F}[g(x)]=G(k)$として、次の等式を証明せよ.\\
\begin{description}
\item[(1)] \hspace{0.5cm}
$\displaystyle{\int_{-\infty}^{\infty} f(t)G(t){\rm d}t = \int_{-\infty}^{\infty} F(t)g(t){\rm d}t}$

\vspace{80mm}
\item[(2)] \hspace{0.5cm}
$\displaystyle{\mathcal{F}[f(ax)]=\frac{1}{|a|}F(\frac{k}{a})}\qquad (aは実定数)$
\end{description}

\newpage

\item ${\phi}_n(x)~(n=1,2,\cdots )$が区間$[a,b]$で正規直交関数系を作るとき、以下の関係式が成立する.以下の問いに答えよ.
\begin{equation*}
\int_a^b {\phi}_m^*(x){\phi}_n(x){\rm d}x={\delta}_{nm}\qquad ({\delta}_{nm}はクロネッカーのデルタ)
\end{equation*}
\begin{enumerate}
\item[(1)] 関数$f(x)$が${\phi}_n(x)$を用いて、\[f(x)=\sum_{n=1}^{\infty}c_n{\phi}_n(x)\]と級数展開できるとすれば、係数$c_n$は、\[c_n=\int_a^b f(x){\phi}_n(x){\rm d}x\]より決められることを示せ.($c_n$を一般化フーリエ係数という)
\newpage
\item[(2)] 関数$\displaystyle{{\phi}_m(x)=\frac{1}{\sqrt{2\pi}}e^{imx}~(m=0,\pm 1,\pm 2,\cdots )}$は区間$[-\pi ,\pi]$で正規直交関数系を作ることを示せ.
\end{enumerate}




\end{enumerate}
\end{document}






