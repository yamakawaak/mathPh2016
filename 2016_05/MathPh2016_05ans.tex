\documentclass[a4paper,11pt,fleqn]{jarticle}
\usepackage[dvipdfmx]{graphicx}
\usepackage{float}
\usepackage{amsmath}
\usepackage{fancyhdr}

\def \vec#1{\mbox{\boldmath $#1$}} %ベクトルマクロ
\def \bun#1#2{\left(\frac{#1}{#2}\right)} %括弧つき分数マクロ
\def \rot{\nabla \times} %rot
\def \div{\nabla \cdot} %div
\def \intt{\int\!\!\!\int} %2重積分
\def \inttt{\int\!\!\!\int\!\!\!\int} %3重積分

% ページレイアウト
\setlength{\topmargin}{10mm}
  \addtolength{\topmargin}{-1in}
\setlength{\oddsidemargin}{30mm}
  \addtolength{\oddsidemargin}{-1in}
\setlength{\textwidth}{150mm}
\setlength{\textheight}{250mm}
\setlength{\headsep}{2zw}
\setlength{\headheight}{2zw}
\setlength{\topskip}{15mm}
\linespread{1.0}

% サブセクションを1.,2.にする設定
\renewcommand{\thesubsection}{\arabic{subsection}.}

% サブサブセクションを(1),(2)にする設定
\renewcommand{\thesubsubsection}{(\arabic{subsubsection})}

% 大問2の3番目の計算式のラベルを(2.3)にする設定
% 計算式の参照には\eqref{eq:hoge}を使う
\makeatletter
  \renewcommand{\theequation}{\arabic{subsection}.\arabic{equation}}
  \@addtoreset{equation}{subsection}
\pagestyle{fancy}
% ヘッダーの設定
  \lhead[物理数学 2016.04.18]{\leftmark}
  \rhead[\leftmark]{物理数学 2016.04.18}
\renewcommand{\headrulewidth}{0pt}
\makeatother



\begin{document}


\begin{center}
\begin{Large}
演習問題その5 解答例
\end{Large}
\end{center}

\subsection{}
次の行列Aの行列式を解け。
\subsubsection{}
\begin{equation*}
A=
\begin{pmatrix}
1 &4 &2 &3 \\
8 &2 &1 &0 \\
2 &4 &7 &1 \\
1 &0 &0 &1
\end{pmatrix}
\end{equation*}
(解答例)\\
余因子展開しやすくなるように、行列式を基本変形していく。
\begin{eqnarray*}
|A|=
\begin{vmatrix}
1 &4 &2 &3 \\
8 &2 &1 &0 \\
2 &4 &7 &1 \\
1 &0 &0 &1
\end{vmatrix}
&~=~& \begin{vmatrix}
1 &4 &2 &3 \\
0 &2 &1 &-8 \\
0 &4 &7 &-1 \\
0 &-4 &-2 &-2
\end{vmatrix}
~=~ \begin{vmatrix}
2 &1 &-8 \\
4 &7 &-1 \\
-4 &-2 &-2
\end{vmatrix}\\
&~=~& \begin{vmatrix}
2 &1 &-8 \\
0 &5 &15 \\
0 &0 &-18
\end{vmatrix}\\
&~=~& 2\times \begin{vmatrix}
5 &15 \\
0 &-18
\end{vmatrix} ~=~ -180.
\end{eqnarray*}


\subsubsection{}
\begin{equation*}
A=
\begin{pmatrix}
a &0 &3a \\
ab &2b &3ab \\
4 &3a^2 &12
\end{pmatrix}
\end{equation*}
(解答例)\\
$(1列)-(3列)\times (1/3)$の基本変形を行うと、
\begin{eqnarray*}
|A|~=~
\begin{vmatrix}
a &0 &3a \\
ab &2b &3ab \\
4 &3a^2 &12
\end{vmatrix}
~=~\begin{vmatrix}
0 &0 &3a \\
0 &2b &3ab \\
0 &3a^2 &12
\end{vmatrix} ~=~ 0.
\end{eqnarray*}


\newpage
\subsection{}
次の行列Aの固有値、固有ベクトルを求めよ。
\begin{equation*}
A=
\begin{pmatrix}
3 &1 \\
2 &2
\end{pmatrix}
\end{equation*}
(解答例)\\
2次の単位行列と求める固有値をそれぞれ$E,\lambda$として固有方程式$|A-\lambda E|=0$を解くと、
\begin{eqnarray*}
|A-\lambda E|~=~
\begin{pmatrix}
3-\lambda &1 \\
2 &2-\lambda
\end{pmatrix}
~=~{\lambda}^2-5\lambda +4~=~0\\
よって、~\lambda ~=~4,~1
\end{eqnarray*}
対応する固有ベクトルは$(A-\lambda E)\vec{x}=\vec{0}$となる$\vec{x}=(x_1,x_2)$を求めればよい。
\begin{itemize}
\item $\lambda =4$のとき、
\begin{eqnarray*}
(A-4E)\vec{x}~=~\begin{pmatrix}
-1 &1 \\
2 &-2
\end{pmatrix}\vec{x}~=~\vec{0}\\
よって、x_1+x_2=0
\end{eqnarray*}
これを満たすすべての$\vec{x}$が固有ベクトル。ここで$x_1=\alpha$とおけば、$x_2=\alpha$となり求める固有ベクトルは以下のように表せられる(ただし$\alpha$は任意の定数)。
\begin{eqnarray*}
\vec{x}~=~\alpha \begin{pmatrix}
1\\
1
\end{pmatrix}.
\end{eqnarray*}
\item $\lambda =1$のとき、
\begin{eqnarray*}
(A-E)\vec{x}~=~\begin{pmatrix}
2 &1 \\
2 &1
\end{pmatrix}\vec{x}~=~\vec{0}\\
よって、2x_1+x_2=0
\end{eqnarray*}
$x_1=\alpha$とおけば、$x_2=-2\alpha$となり求める固有ベクトルは以下のように表せられる(ただし$\alpha$は任意の定数)。
\begin{eqnarray*}
\vec{x}~=~\alpha \begin{pmatrix}
1\\
-2
\end{pmatrix}.
\end{eqnarray*}
\end{itemize}

\newpage
\subsection{}
以下のヤコビアンを求めよ。
\subsubsection{}
$$u ~=~ x+y+z,$$
$$v ~=~ 3x+4y+8z,$$
$$w ~=~ 2x+2y+z,$$
$~~~~$のヤコビアン$\partial (u,v,w)/\partial (x,y,z)$.

\begin{eqnarray*}
(解答例)
\end{eqnarray*}

\begin{eqnarray*}
\frac{\partial (u, v, w)}{\partial (x, y, z)}&=& 
\begin{array}{|ccc|}
\frac{\partial u}{\partial x} & \frac{\partial u}{\partial y} & \frac{\partial u}{\partial z} \\
\frac{\partial v}{\partial x} & \frac{\partial v}{\partial y} & \frac{\partial v}{\partial z} \\
\frac{\partial w}{\partial x} & \frac{\partial w}{\partial y} & \frac{\partial w}{\partial z}
\end{array}
=
\begin{array}{|ccc|}
1&1&1\\
3&4&8\\
2&2&1
\end{array}
\\
&&\\
&=&4+6+16-(8+3+16)\\
&=&-1.
\end{eqnarray*}

\newpage
\subsubsection{$a,b,cを定数とするとき、$}
$$x~=~u+v+w,$$
$$y~=~au+bv+cw,$$
$$z~=~a^2u+b^2v+c^2w,$$
$~~~~$のヤコビアン$\partial (x,y,z)/\partial (u,v,w)$.

\begin{eqnarray*}
(解答例)
\end{eqnarray*}

\begin{eqnarray*}
\frac{\partial (x, y, z)}{\partial (u, v, w)}&=&
\begin{array}{|ccc|}
\frac{\partial x}{\partial u} & \frac{\partial x}{\partial v} & \frac{\partial x}{\partial w} \\
\frac{\partial y}{\partial u} & \frac{\partial y}{\partial v} & \frac{\partial y}{\partial w} \\
\frac{\partial z}{\partial u} & \frac{\partial z}{\partial v} & \frac{\partial z}{\partial w}
\end{array}
=
\begin{array}{|ccc|}
1&1&1\\
a&b&c\\
a^2&b^2&c^2\\
\end{array}
\\
&&\\
&=& ab^2+bc^2+ca^2-(a^2b+b^2c+c^2a)\\
&=&-ab(a-b)+c(a^2-b^2)-c^2(a-b)\\
&=& (a-b)(-ab+bc+ca-c^2)\\
&=& (a-b)\{b(c-a)-c(c-a)\}\\
&=& (a-b)(b-c)(c-a).
\end{eqnarray*}

\newpage
\subsection{}
円柱座標系$(\rho ,\phi ,z)$、球座標系$(r,\theta ,\phi)$での微小体積$dV$をヤコビアンの計算から求めよ。
\begin{eqnarray*}
(解答例)
\end{eqnarray*}
\begin{itemize}
\item 円柱座標系
\begin{eqnarray*}
x=\rho\cos\phi ,~~y=\rho\sin\phi ,~~z=z
\end{eqnarray*}
であり、微小体積は$dV=\left| \frac{\partial (x,y,z)}{\partial (\rho ,\phi ,z)}\right| d\rho d\phi dz$であるから、まずヤコビアンを計算する。
\begin{eqnarray*}
\frac{\partial (x, y, z)}{\partial (\rho , \phi , z)}&=&
\begin{vmatrix}
\frac{\partial x}{\partial \rho}  &\frac{\partial x}{\partial \phi}  &\frac{\partial x}{\partial z} \\
\frac{\partial y}{\partial \rho}  &\frac{\partial y}{\partial \phi}  &\frac{\partial y}{\partial z} \\
\frac{\partial z}{\partial \rho}  &\frac{\partial z}{\partial \phi}  &\frac{\partial z}{\partial z}
\end{vmatrix}
\\&=&
\begin{vmatrix}
\cos\phi &-\rho\sin\phi &0 \\
\sin\phi &\rho\cos\phi &0 \\
0 &0 &1
\end{vmatrix}
\\&=&\rho d\rho d\phi dz
\end{eqnarray*}
よって微小体積は、$dV=|\rho d\rho d\phi dz|=\rho d\rho d\phi dz$.

\item 球座標系
\begin{eqnarray*}
x=r\sin\theta\cos\phi ,~~y=r\sin\theta\sin\phi ,~~z=r\cos\theta
\end{eqnarray*}
であり、微小体積は$dV=\left| \frac{\partial (x,y,z)}{\partial (r ,\theta ,\phi )}\right| dr d\theta d\phi$であるから、まずヤコビアンを計算する。

\begin{eqnarray*}
\frac{\partial (x, y, z)}{\partial (r , \theta , \phi )}&=&
\begin{vmatrix}
\frac{\partial x}{\partial r}  &\frac{\partial x}{\partial \theta}  &\frac{\partial x}{\partial \phi} \\
\frac{\partial y}{\partial r}  &\frac{\partial y}{\partial \theta}  &\frac{\partial y}{\partial \phi} \\
\frac{\partial z}{\partial r}  &\frac{\partial z}{\partial \theta}  &\frac{\partial z}{\partial \phi}
\end{vmatrix}
\\&=&
\begin{vmatrix}
\sin\theta\cos\phi &r\cos\theta\cos\phi &-r\sin\theta\sin\phi \\
\sin\theta\sin\phi &r\cos\theta\sin\phi &r\sin\theta\cos\phi \\
\cos\theta &-r\sin\theta &0
\end{vmatrix}
\\&=&-r^2\sin\theta drd\theta d\phi
\end{eqnarray*}
よって微小体積は、$dV=|-r^2\sin\theta drd\theta d\phi |=r^2\sin\theta drd\theta d\phi$.
\end{itemize}


\newpage
\subsection{}
楕円体$\frac{x^2}{a^2}+\frac{y^2}{b^2}+\frac{z^2}{c^2}=r^2$中の位置ベクトルは、球座標系を用いて
$$x~=~ar\sin\theta\cos\phi$$
$$y~=~br\sin\theta\sin\phi$$
$$z~=~cr\cos\theta$$
で表される。微小体積を積分することによって、この楕円体の体積を求めよ。\\
\begin{eqnarray*}
(解答例)
\end{eqnarray*}
微小体積は$dV=\left| \frac{\partial (x,y,z)}{\partial (r ,\theta ,\phi )}\right| dr d\theta d\phi$であるから、まずヤコビアンを計算する。

\begin{eqnarray*}
\frac{\partial (x, y, z)}{\partial (r , \theta , \phi )}=
\begin{vmatrix}
\frac{\partial x}{\partial r}  &\frac{\partial x}{\partial \theta}  &\frac{\partial x}{\partial \phi} \\
\frac{\partial y}{\partial r}  &\frac{\partial y}{\partial \theta}  &\frac{\partial y}{\partial \phi} \\
\frac{\partial z}{\partial r}  &\frac{\partial z}{\partial \theta}  &\frac{\partial z}{\partial \phi}
\end{vmatrix}
=
\begin{vmatrix}
a\sin\theta\cos\phi &ar\cos\theta\cos\phi &-ar\sin\theta\sin\phi \\
b\sin\theta\sin\phi &br\cos\theta\sin\phi &br\sin\theta\cos\phi \\
c\cos\theta &-cr\sin\theta &0
\end{vmatrix}
\end{eqnarray*}
よって微小体積は、$dV=|-abcr^2\sin\theta drd\theta d\phi |=abcr^2\sin\theta drd\theta d\phi$.
\\
もとめる体積は、
\begin{eqnarray*}
V &~=~& \int_0^1 \int_0^{2\pi} \int_0^{\pi}abcr^2\sin\theta drd\theta d\phi \\
&~=~& \frac{4\pi}{3}abc
\end{eqnarray*}.

\end{document}
