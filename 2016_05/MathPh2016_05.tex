\documentclass[a4paper,11pt,fleqn]{jarticle}
\usepackage[dvipdfmx]{graphicx}
\usepackage{float}
\usepackage{amsmath}
\usepackage{fancyhdr}

\def \vec#1{\mbox{\boldmath $#1$}} %ベクトルマクロ
\def \bun#1#2{\left(\frac{#1}{#2}\right)} %括弧つき分数マクロ
\def \rot{\nabla \times} %rot
\def \div{\nabla \cdot} %div
\def \intt{\int\!\!\!\int} %2重積分
\def \inttt{\int\!\!\!\int\!\!\!\int} %3重積分

% ページレイアウト
\setlength{\topmargin}{10mm}
  \addtolength{\topmargin}{-1in}
\setlength{\oddsidemargin}{30mm}
  \addtolength{\oddsidemargin}{-1in}
\setlength{\textwidth}{150mm}
\setlength{\textheight}{250mm}
\setlength{\headsep}{2zw}
\setlength{\headheight}{2zw}
\setlength{\topskip}{15mm}
\linespread{1.0}

% サブセクションを1.,2.にする設定
\renewcommand{\thesubsection}{\arabic{subsection}.}

% サブサブセクションを(1),(2)にする設定
\renewcommand{\thesubsubsection}{(\arabic{subsubsection})}

% 大問2の3番目の計算式のラベルを(2.3)にする設定
% 計算式の参照には\eqref{eq:hoge}を使う
\makeatletter
  \renewcommand{\theequation}{\arabic{subsection}.\arabic{equation}}
  \@addtoreset{equation}{subsection}
\pagestyle{fancy}
% ヘッダーの設定
  \lhead[物理数学 2016.04.18]{\leftmark}
  \rhead[\leftmark]{物理数学 2016.04.18}
\renewcommand{\headrulewidth}{0pt}
\makeatother



\begin{document}


\begin{center}
\begin{Large}
演習問題その5 行列式・ヤコビアン
\end{Large}
\end{center}

\subsection{}
次の行列Aの行列式を解け。
\subsubsection{}
\begin{equation*}
A=
\begin{pmatrix}
1 &4 &2 &3 \\
8 &2 &1 &0 \\
2 &4 &7 &1 \\
1 &0 &0 &1
\end{pmatrix}
\end{equation*}

\vspace{60mm}
\subsubsection{}
\begin{equation*}
A=
\begin{pmatrix}
a &0 &3a \\
ab &2b &3ab \\
4 &3a^2 &12
\end{pmatrix}
\end{equation*}

\newpage
\subsection{}
次の行列Aの固有値、固有ベクトルを求めよ。
\begin{equation*}
A=
\begin{pmatrix}
3 &1 \\
2 &2
\end{pmatrix}
\end{equation*}

\newpage
\subsection{}
以下のヤコビアンを求めよ。
\subsubsection{}
$$u ~=~ x+y+z,$$
$$v ~=~ 3x+4y+8z,$$
$$w ~=~ 2x+2y+z,$$
$~~~~$のヤコビアン$\partial (u,v,w)/\partial (x,y,z)$.

\newpage
\subsubsection{$a,b,cを定数とするとき、$}
$$x~=~u+v+w,$$
$$y~=~au+bv+cw,$$
$$z~=~a^2u+b^2v+c^2w,$$
$~~~~$のヤコビアン$\partial (x,y,z)/\partial (u,v,w)$.

\newpage
\subsection{}
円柱座標系$(\rho ,\phi ,z)$、球座標系$(r,\theta ,\phi)$での微小体積$dV$をヤコビアンの計算から求めよ。

\newpage
\subsection{}
楕円体$\frac{x^2}{a^2}+\frac{y^2}{b^2}+\frac{z^2}{c^2}=r^2$中の位置ベクトルは、球座標系を用いて
$$x~=~ar\sin\theta\cos\phi$$
$$y~=~br\sin\theta\sin\phi$$
$$z~=~cr\cos\theta$$
で表される。微小体積を積分することによって、この楕円体の体積を求めよ。\\
ヒント:積分区間は$0\leq r\leq 1,~0\leq\theta\leq \pi ,~0\leq\phi\leq 2\pi$となる。


\end{document}