\documentclass[11pt]{jarticle}
\setlength{\textheight}{43\baselineskip}
\addtolength{\textheight}{\topskip}
\setlength{\textheight}{21\footskip}
\setlength{\textwidth}{42zw}
\setlength{\hoffset}{-5zw}
\usepackage{amsmath,amssymb,bm}
%
\usepackage{fancyhdr} %ヘッダー
\pagestyle{fancy}
\rhead{物理数学 \ 2016.05.02}
\renewcommand{\headrulewidth}{0pt} %ヘッダーラインを打ち出さない
%
\def \bun#1#2{\left(\frac{#1}{#2}\right)} %括弧付き分数マクロ
\def \vec#1{\mbox{\boldmath $#1$}} %ベクトルマクロ
\def \rot{\nabla \times} %ローテーション
\def \div{\nabla \cdot} %ダイバージェンス
\def \intt{\int\!\!\!\int} %2重積分
\def \intf#1#2#3#4{\int_{#1}^{#2}\!\!\!\int_{#3}^{#4}} %2重積分範囲付き
\def \inttt{\int\!\!\!\int\!\!\!\int} %3重積分
\def \intff#1#2#3#4#5#6{\int_{#1}^{#2}\!\!\!\int_{#3}^{#4}\!\!\!\int_{#5}^{#6}} %3重積分範囲付き
%
\usepackage{graphicx}
%%------------------------------%%
\begin{document}
\begin{center}
{\Large
演習問題その8小テスト}\\
\ \\
\underline{学籍番号:          }, 
\underline{氏名:            }
\end{center}
\begin{enumerate}
%%-----(問題開始)-----%%
\item[1.]
次の非斉次微分方程式の一般解を求めよ.
\begin{eqnarray*}
y''-6y'+9y=\sin{x}
\end{eqnarray*}


%%-----(問題終了)-----%%
\end{enumerate}


\newpage
%%------------------------------%%
%%-----(   解答ここから   )-----%%
\begin{center}
{\Large
演習問題その8小テスト 解答例}\\
\end{center}
\begin{enumerate}
\item[1.]
次の非斉次微分方程式の一般解を求めよ.
\begin{eqnarray*}
y''-6y'+9y=\sin{x}
\end{eqnarray*}

\begin{eqnarray*}
(解答例)
\end{eqnarray*}
まず斉次方程式に$y_0=e^{kx}$を代入すると、特性方程式$k^2-6k-9=0$を得る。
これを解くと、$k=3$(重解)となるので、斉次方程式の一般解は
\begin{eqnarray*}
y_0=C_1e^{3x}+C_2xe^{3x}
\end{eqnarray*}
となる。次に特解の形を$y_p=A\sin{x}+B\cos{x}$として代入すると、$A=4/50, B=3/50$と求まる。
よってもとの微分方程式の一般解は
\begin{eqnarray*}
y=y_0+y_p=C_1e^{3x}+C_2xe^{3x}+\frac{4\sin{x}+3\cos{x}}{50}
\end{eqnarray*}

%%-----(解答終了)-----%%
\end{enumerate}

\newpage
(採点基準)
10点満点。\\
斉次解が求められていれば+6点、特殊解が求められていれば+4点とした。\\
そのほかオマケで加点してある。\\
特殊解は、本問の場合$y_p=A\cos x+B\sin x$とおいて、係数$A,B$を求めると良い。
\end{document}
