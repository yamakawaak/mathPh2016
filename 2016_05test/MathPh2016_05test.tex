\documentclass[11pt]{jarticle}
\setlength{\textheight}{43\baselineskip}
\addtolength{\textheight}{\topskip}
\setlength{\textheight}{21\footskip}
\setlength{\textwidth}{42zw}
\setlength{\hoffset}{-5zw}
\usepackage{amsmath,amssymb,bm}
%
\usepackage{fancyhdr} %ヘッダー
\pagestyle{fancy}
\rhead{物理数学 \ 2016.04.25}
\renewcommand{\headrulewidth}{0pt} %ヘッダーラインを打ち出さない
%
\def \bun#1#2{\left(\frac{#1}{#2}\right)} %括弧付き分数マクロ
\def \vec#1{\mbox{\boldmath $#1$}} %ベクトルマクロ
\def \rot{\nabla \times} %ローテーション
\def \div{\nabla \cdot} %ダイバージェンス
\def \intt{\int\!\!\!\int} %2重積分
\def \intf#1#2#3#4{\int_{#1}^{#2}\!\!\!\int_{#3}^{#4}} %2重積分範囲付き
\def \inttt{\int\!\!\!\int\!\!\!\int} %3重積分
\def \intff#1#2#3#4#5#6{\int_{#1}^{#2}\!\!\!\int_{#3}^{#4}\!\!\!\int_{#5}^{#6}} %3重積分範囲付き
%
\usepackage{graphicx}
%%------------------------------%%
\begin{document}
\begin{center}
{\Large
演習問題その5小テスト}\\
\ \\
\underline{学籍番号:          }, 
\underline{氏名:            }
\end{center}
\begin{enumerate}
%%-----(問題開始)-----%%
\item[1.]
円柱座標系$(\rho ,\phi ,z)$から球座標系$(r,\theta ,\phi )$への変換に対するヤコビアンを求めよ.
%%-----(問題終了)-----%%
\end{enumerate}


\newpage
%%------------------------------%%
%%-----(   解答ここから   )-----%%
\begin{center}
{\Large
演習問題その5小テスト 解答例}\\
\end{center}
\begin{enumerate}
\item[1.]
円柱座標系$(\rho ,\phi ,z)$から球座標系$(r,\theta ,\phi )$への変換に対するヤコビアンを求めよ.
\\
(解答例)\\
\begin{eqnarray*}
\left\{
\begin{array}{cl}
\rho=&r\sin\theta \\
\phi=&\phi \\
z=&r\cos\theta
\end{array}
\right.
\end{eqnarray*}
であるから、ヤコビアンは
\begin{eqnarray*}
\frac{\partial(\rho,\phi,z)}{\partial(r,\theta,\phi)}&=&
\begin{array}{|ccc|}
\sin\theta & r\cos\theta & 0 \\
0 & 0 & 1 \\
\cos\theta & -r\sin\theta & 0
\end{array}
\\[10pt]
&=&
(-1)^{(3+2)} \;
\begin{array}{|cc|}
\sin\theta &r\cos\theta\\
\cos\theta &-r\sin\theta
\end{array}
\\
\\
&=& r.
\end{eqnarray*}

%%-----(解答終了)-----%%
\end{enumerate}

\newpage
採点基準\\
\\
$(\rho ,\phi ,z)$を$(r,\theta ,\phi)$で書き換えられている:5点\\
\\
ヤコビアンの表式が正しい:2点\\
\\
最終的な答えが正しい:3点\\
\\
$\frac{\partial (x,y,z)}{\partial (\rho ,\phi ,z)},~\frac{\partial (x,y,z)}{\partial (r,\theta ,\phi )}$が書けている場合は、おまけでそれぞれ+1点\\
\\
答えを$\frac{r^2\sin\theta}{\rho}$としているものはそれぞれの座標系の変数が混在しているため、8点とした。

\end{document}
