\documentclass[11pt]{jsarticle}
\setlength{\textheight}{43\baselineskip}
\addtolength{\textheight}{\topskip}
\setlength{\textheight}{21\footskip}
\setlength{\textwidth}{42zw}
%\setlength{\hoffset}{-5zw}
\usepackage{amsmath,amssymb,bm}
%\usepackage{txfonts}  %定義の記号
\usepackage{fancyhdr} %ヘッダー
\pagestyle{fancy}
\rhead{物理数学演習 \ 2016/05/26}
\renewcommand{\headrulewidth}{0pt} %ヘッダーラインを打ち出さない

\def \bun#1#2{\left(\frac{#1}{#2}\right)} %括弧付き分数マクロ
\def \vec#1{\mbox{\boldmath $#1$}} %ベクトルマクロ
\def \rot{\nabla \times} %ローテーション
\def \div{\nabla \cdot} %ダイバージェンス
\def \intt{\int\!\!\!\int} %2重積分
\def \intf#1#2#3#4{\int_{#1}^{#2}\!\!\!\int_{#3}^{#4}} %2重積分範囲付き
\def \inttt{\int\!\!\!\int\!\!\!\int} %3重積分
\def \intff#1#2#3#4#5#6{\int_{#1}^{#2}\!\!\!\int_{#3}^{#4}\!\!\!\int_{#5}^{#6}} %3重積分範囲付き

\usepackage{graphicx}
\usepackage{wrapfig}%テキスト回り込み

\def \Vec#1{\mbox{\boldmath $#1$}} %ベクトルマクロ
\begin{document}
\begin{center}
{\Large 演習問題その13 小テスト
}\\
\ \\
\underline{学籍番号:          }, 
\underline{氏名:            }
\end{center}
\begin{enumerate}
%%------------------------------%%
\item $\displaystyle{\frac{\partial u }{\partial x} = x+y}$
を満たし,$x=0$で$u=e^{-y}$に従って指数関数的に減少する解を求めよ.

%%------------------------------%%
\end{enumerate}
%%-----(問題ここまで)-----%%







\newpage
%%-----(解 答  開 始)-----%%
\setcounter{page}{1}
\begin{center}
{\Large
演習問題その13小テスト 解答例}\\
\end{center}
\begin{enumerate}
%%------------------------------%%
\item $\displaystyle{\frac{\partial u }{\partial x} = x+y}$
を満たし,$x=0$で$u=e^{-y}$に従って指数関数的に減少する解を求めよ.\\
\\
(解答例)\\
\begin{eqnarray*}
&&\frac{\partial u }{\partial x} = x+y \\[5pt]
&\Leftrightarrow& \ u = \frac{1}{2}x^2 + xy + f(y).
\end{eqnarray*}
$x=0$における条件式を代入すると,
\begin{eqnarray*}
&& e^{-y} = f(y). \\[5pt]
&\therefore& u = \frac{1}{2}x^2+ xy + e^{-y}.
\end{eqnarray*}

%%-----(解答ここまで)-----%%
\end{enumerate}
\end{document}

%%%%%%%%%%%
